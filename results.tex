\chapter{EVALUATION}
\label{chapter:EVALUATION}
\section{Overview:}
The purpose of this section is to provide a broad overview of the major findings of the thesis, discuss their importance and answer the research questions raised in Chapter 1. The chapter evaluates and illustrate how the customized feedback system was developed and deployed to provide feedback hints to the students after submitting their JUnit Test solutions on the system. The PiTest is the tool that I used to generate mutants. The outcome of this study have shown that my newly developed customized feedback system was able to effectively provide meaningful and customizable feedback hints to early learners with a basic understanding of JUnit Testing skills. When my customized feedback system is compared to the PiTest feedback system, my feedback system offers better results since its feedback hints explain to the students where they went wrong and how to rectify their mistakes. 

\section{ Evaluation Process}
In this study, 45 students were contacted to participate in the research. Of them, 25 students showed interest to participate in the research while 20 students did not show interest. Of the 25 students that showed interest only 14 of them completed their assignments and submitted their solutions successfully while 11 students failed to submit their solutions. The student participants were provided with assignments of a Java Application named StudentBankingAppAssignment, which was a banking demo application, with functionalities to withdraw money, pay in money and check balance. The students who agreed to participate in the research were required to complete the assignments and submit their solutions to the proposed feedback system. \\
\\
The students were then given more instructions, following which they were to carry out the tasks and submit their solutions. Additionally, I had to provide them instructions of how to name the project, which was provided in a folder that was stored in the drop box. The students were requested to provide a valid email address as the last piece of instruction. Projects processing, Projects failure, Projects Success, Projects Submitted, Projects Incorrect, project assignment was among the folders that could be found in the drop box. After then, the students had two weeks to finish the assignment and submit it via the feedback system. I chose to give the students two weeks so they would have enough time to finish the task, considering that the students may have had other school courses with assignments that they had to complete. \\
\\ 
The students that took part in the evaluation procedure ranged in age from 20 to 27 and were from different universities and colleges but they all belonged to the Computer Science Departments of their respective institutions. According to their feedback, the students who participated volunteered to participate to learn the fundamentals of JUnit testing as well as gain a better understanding of how advanced JUnit testing was carried out in programming assignments while others pointed out that their participation had to do with their passion in programming studies. Some of the students who participated had advanced levels of JUnit testing skills, while others were early learners. The students also had the choice of delivering their work as a Zip file or as a folder containing all the files required to complete their project.\\
\\ 
Students received assistance by being given resources such as tutorial videos explaining to the related assignments, as well as samples of assignment solutions that had passed the basic stage of JUnit testing. This was done to make sure that the students understood what lay ahead of them as far as the assignment is concerned as well as what was expected of them when writing code for their project solutions.  Additionally, the students were supposed to contact a given email address if they had any queries or encounter difficulties in the task they were assigned. The email was meant to be used for requesting help. It's also important to remember that the students that took part in this exercise were from colleges and universities located all over the continents of Africa and Europe.\\
\\
The first step was for the students to access the assignment using a Dropbox folder labeled " project assignment." It is significant to note that not all of the students who were contacted agreed to take part in this evaluation procedure. The solutions to these students' assignments were not entered into the system because they either did not respond when assignments were given to them in the " project assignment" folder or did not follow the instructions that were given to them. While the solutions provided by a different group of students who took part in the assessment were successfully implemented, it was later discovered that these solutions contained errors that prevented the system from processing them further. As a result, these projects were moved into the "projects failure" folder.\\
\\ 
The instructor, myself, determined the root causes of their errors and afterward analyzed the projects that were transferred to the “projects failure” folder in further detail. Another set of students also submitted valid solutions, however, some of these valid solutions did not follow the initial instructions, and as a consequence, they were unable to finish the procedure; as a result, the system transferred them to a folder named "projects incorrect format."\\
\\ 

After the students successfully finished their projects, they had the option of providing their feedback or opinion based on their utilization of the developed feedback system as a whole and how well the assignments were handled. The feedback from the students was mainly positive, as they found that using the proposed feedback system had helped them to understand how advanced JUnit testing was carried out in programming assignments. Many of the students commented on how useful the tutorials and sample solutions had been for giving them an understanding of what was expected when writing code for their own project solutions. Additionally, many of the participants expressed appreciation for being given two weeks to complete their assignments and submit their solutions, as it allowed them enough time to focus on other school coursework while still having enough time to finish their assignments. \\

In terms of negative feedback, some of the participants noted that they would have appreciated more detailed instructions regarding naming conventions for their projects. Others also pointed out that they would have liked more clarification on certain aspects of the assignment such as which libraries were required and what specific types of code snippets were needed for certain functions. Some students also noted that they would have liked more resources available in terms of tutorials or sample solutions. \\

Most of the participants found that using the proposed feedback system was helpful and provided them with a better understanding of how JUnit testing worked in programming assignments. The majority of comments related to issues with instructions or lack of resources, which could be addressed by providing additional guidance or making more resources available online. \\

I evaluated the feedback they provided that was saved in a Dropbox folder called "student feedback opinion." More so this students’ feedback were written in text files. However, it is important to note that the students’ comments and opinions were given anonymously because no personal information regarding the participants was required for this evaluation to avoid data privacy violation. Additionally, the main reason why the feedback was delivered in written text files is just for simplicity’s sake. It was necessary to make it easy for students to provide their feedback of their experience after using the proposed feedback system. \\

% The findings from this research indicate that providing customized feedback may be beneficial for early learners of Object-Oriented Programming as it can help them debug their programs more quickly and easily. Additionally, it provides insight into potential areas of improvement in their solutions, which can help them become better programmers in the long run. Furthermore, utilizing mutation testing tools such as PiTest can help improve the test suites used to evaluate student solutions, leading to more accurate assessment results.\par 
% A customized feedback system aimed at helping students to improve their JUnit testing skills is an important factor in learning. Many tools that offer programming exercises provide automated feedback on student solutions. However, these automated feedbacks do not provide meaningful feedback systems that have solved equations of tests that didn't go through. The system is designed to have features that help classify the feedback hints as something early learners can understand. The techniques used to generate feedback are adaptable and easily evaluated. The system’s time scheduler is advantageous to the students as it helps them set the timers that they prefer. Furthermore, teachers cannot easily adapt schedules of their own as they issue tests to students.\par 
% Given the role of feedback in learning, the development of this customized feedback system wanted to come up with a feedback system that not only helped students with advanced knowledge of JUnit testing but also beginners who had a basic knowledge of JUnit testing skills. The feedback system has narrowed the scope by only considering tools that offer customized feedback hints that let early learners practice by carrying out mutation testing on their own.\par 
% There is a growing body of research on tools for the provision of feedback to students’ assignments the findings of this thesis are that customized feedback is not only limited to early learners but also anyone interested in coding and would like to learn more. Although many feedback systems have been developed, most of them do not include tools that enable students to rectify their mistakes but rather general feedback which might not be helpful for beginner students who only have the basic knowledge. The rationale of my functionality criteria is that the ability to develop a program to solve a particular problem is an important learning objective for learning programming. \par 
% It was discovered that a substantial correlation between the degree to which students engage with the system under test and their impression of the usefulness of the customized feedback system. Another finding was that when assignments are included in the feedback systems curriculum, they effectively play a key role in on how beginner students adapt to programming courses.  
% Since my main interest is to improve learning, I focus on customized feedback. I used the domain criteria to focus my review on programming languages used in the industry or taught at universities. The type of JUnit test that supports this feedback system determines to a large extent how difficult it is to generate feedback. The solutions submitted by the students in the customized feedback system are often quiz-like questions with a single solution, in a program that needs to undergo the who feedback system so that meaningful feedback hints can be sent to the students. During the evaluation stages, students can receive feedback if their solution failed to pass a particular stage. One advantage of this system is that students are allowed to rectify their mistakes and resubmit the solution. Usually a program skeleton or other information that suggests the solution strategy is provided, but variations in the implementation are allowed. I select papers and tools that satisfy all inclusion criteria and none of the exclusion criteria. Since no review addressing my research questions has been conducted before, my customized feedback system has a complete overview of the field.
\section{Importance of the findings}

The data collected from my proposed feedback system with mutation testing shows that it is both simple and provides meaningful feedback. I had carefully categorized and organized the codebase into packages and folders, which made the system easier to use and understand for tutors and professors. Additionally, I used Dropbox to store project files submitted by students, which increased the flexibility of the system. \par

The evaluation process revealed that the student participants responded positively to the system and found it easy to use while providing meaningful feedback. The students felt that they could finish their assignments quickly due to the detailed instructions provided by me as well as the tutorial videos which helped them understand better how JUnit testing was conducted in programming assignments. Furthermore, they also appreciated the fact that they could submit their solutions as a Zip file or folder containing all the files required for their assignment which made it easier for them to complete their tasks. \par

The advantages of using JUnit tests in code were also demonstrated during this evaluation process. Firstly, writing unit tests helps create a robust system capable of responding gracefully to unforeseen events and user errors. Secondly, JUnit tests can help identify coding bugs early in the development process and provide meaningful feedback that helps developers fix these issues quickly and accurately. Thirdly, unit tests can be used for refactoring code, as they will quickly identify any unintended changes that may occur due to design modifications. Finally, JUnit tests provide a layer of documentation for the codebase making it easier for developers to understand the structure and purpose of each component. 
Having a good test coverage also ensures improved quality of software by helping us validate if our application is free of defects and if it is working as per our expectations. Furthermore, it makes the codebase easy to maintain and reusable so anyone can reference them and execute the test cases in the future. \par

Overall, my data shows that my proposed feedback system was indeed simple and provided meaningful feedback which helped students improve their programming skills quickly and accurately. The use of JUnit testing was key in this process as it enabled me to detect bugs early on in the development process and ensured that my system was robust enough to handle unforeseen events or user errors. All these factors combined resulted in meaningful feedback being generated for programming assignments with mutation testing, thus providing a better understanding result when compared to traditional methods such as manual reviews or peer grading systems. \par


\section{Research questions:}
\begin{enumerate}
\item  What type of auto-generated feedback hints is useful for early learners of OOP with basic knowledge of Unit Testing? 
The first research question focused on determining what type of auto-generated feedback hints would be useful for early learners of OOP with basic knowledge of Unit Testing. The results showed that providing customized feedback hints that were made easy for students to understand and tailored to the individual student’s solutions was beneficial for early learners. Furthermore, utilizing mutation testing tools such as PiTest can help improve the test suites used to evaluate student solutions, leading to more accurate assessment results.
\item  How can the proposed feedback system be fully automated to generate meaningful feedback hints to early learners of OOP with basic knowledge of JUnit testing?
The second research question focused on how the proposed feedback system could be fully automated to generate meaningful feedback hints to early learners of OOP with basic knowledge of JUnit testing. The results indicated that the newly developed feedback system was able to effectively provide meaningful and customizable feedback hints to early learners of Object-Oriented Programming. The feedback system was developed using the PiTest mutation testing tool, which enabled the evaluation of the quality of the student’s submissions. Through this system, students were able to submit their solutions and receive clear and concise feedback hints that were much better than what was generated by PiTest alone.
\end{enumerate}
\\
\\
\\
\\
%\section{ Overview: Summary of the results} 
%The results of this study showed that the newly developed feedback system was able to effectively provide meaningful and customizable feedback hints to early learners of Object-Oriented Programming. The feedback system was developed using the PiTest mutation testing tool, which enabled the evaluation of the quality of the student's submissions. Through this system, students were able to submit their solutions and receive clear and concise feedback hints that were much better than what was generated by PiTest alone. Furthermore, the feedback messages were made easy for students to understand and were provided directly to their emails for review.
%\section{Main findings: Report relevant results concisely and objectively, in a logical order}
%I selected a group of students who had a basic understanding of JUnit Testing skills as test subjects and subjected them to this experiment. Each student was provided with JUnit tests which they were required to solve and submit their solutions to the customized feedback system. Upon their submission, the system moved the submitted solutions from the submitted directory to the working directory and then moved on to the next stage till it provided the customized feedback as explained in “Implementation.”
%The findings from this research indicate that providing customized feedback may be beneficial for early learners of Object-Oriented Programming as it can help them debug their programs more quickly and easily. Additionally, it provides insight into potential areas of improvement in their solutions, which can help them become better programmers in the long run. Furthermore, utilizing mutation testing tools such as PiTest can help improve the test suites used to evaluate student solutions, leading to more accurate assessment results.
%\section{Research questions: [What makes customized feedback} better? What benefits early learners? What helps debug programs? What uses mutation testing to improve the test suite?]
%This research investigated what makes customized feedback better for early learners of Object-Oriented Programming. The findings suggest that providing customized feedback can be beneficial for students in several ways. Firstly, it can help students debug their programs more quickly and easily by providing specific hints as to where their solutions may have gone wrong. Additionally, it can provide insight into potential areas of improvement in their solutions, thereby helping them become better programmers in the long run. Finally, by utilizing mutation testing tools such as PiTest to improve the test suites used to evaluate student solutions, more accurate assessment results can be achieved.
%\noindent\lipsum[15-20]

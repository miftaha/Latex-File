\chapter{Discussion and Future Work}
\label{chapter: Discussion and Future Work}
\section{Summary}
The results of this research indicate that providing customized feedback to early learners of Object-Oriented Programming is beneficial in several ways. Firstly, it can help them identify errors in their code more quickly. Secondly, it provides insight into potential areas of improvement in their solutions, which can help them become better in writing JUint test cases in the long run. Lastly, utilizing mutation testing tools such as PiTest is helpful to analyze the students test suites and to evaluate student solutions, leading to more accurate assessment results.
\section{Interpretations}
Looking at the customized feedback system, the evaluation has shown that giving customized feedback hints to students is important as it helps them identify errors in their code and rectify them. In the first interaction of the system, the PITest tool has come out as an outstanding tool that is compatible with this system because it ensures the quality of the student software testing suite and also efficient in generating mutants. 
\\
\\
Generating feedback is a useful way to point out errors in the students' solutions and emphasizes the importance of mutation testing to programming students. It is therefore a valuable technique, and relatively easy to implement using existing test frameworks. Most tools that use automated testing support the PITest tool, because black-box testing does not require using a specific algorithm.
\\
\\
The only aspect that matters is whether the output meets the requirements of the exercise. However, only giving test-based feedback will not in all cases help a student fix an incorrect program. To help a student improve their JUnit testing skills, the feedback given should explain why the test failed, how the code was supposed to be, and suggestions on how to re-edit the code. Feedback on how to proceed is necessary to both fixing problems and progress toward a solution when the students are facing issues. This type of feedback is mostly seen in the form of error correction feedback hints and much less as hints on task-processing steps as well as program improvements.
\\
\\ 
The findings from this research suggest that providing customizable feedback hints is beneficial for early learners of Object-Oriented Programming because it helps them identify their errors and fix them more quickly and easily. It also provides insight into potential areas for improvement in their solutions. Furthermore, utilizing mutation testing tools such as PiTest can help improve the test suites used to evaluate student solutions and lead to more accurate assessment results. Thus, this type of feedback system could be useful for helping students learn and improve their skills over time. Tests suits that passed mutation testing means they are good. 
\\
\\
The system offers the professor and tutor options to customize the feedback hints using text files format. Using text file format, the professor can specify custom feedback hints and the accepted threshold that the student must obtain to be considered as passed. There is a vast number of studies that identify and classify difficulties and errors that early learners encounter in their learning process. In my case, the customizable feedback hints had better level of feedback satisfaction compared to the Pitest generated feedback. 
\section{Advantages}
The advantages of this study are significant because they suggest that providing meaningful and customizable feedback hints is an effective way to help early learners with basic knowledge of OOP improve their Junit testing skills. This type of system would be especially useful for professors or tutors who are teaching large classes or dealing with a large volume of assignments regularly and due to its automated nature, which reduces the amount of time spent giving individualized feedback for each student's assignment solution.
\\
\\
Many of the previous feedback systems have only been able to deliver generalized feedback. However, my customized feedback system goes further to allowing the professor or tutor give a customized feedback hint that early learners with basic knowledge JUnit testing skills can understand. If students want to learn from their mistakes, they should familiarize themselves with using my feedback system. The use of this system enables lecturers to focus their attention to the students whose code failed. By doing so they can engage the students in person and help them identify their weaknesses and improve on them.
\section{Limitations}

The evaluation process used in this study is subject to several limitations that threaten its validity. The first issue relates to the accuracy and meaningfulness of the variables being measured. In this study, the feedback system was tested on a small sample size of 45 students, 25 of whom agreed to participate. This limited sample size may not accurately represent the wider population, as it is impossible to guarantee that the participants were representative of the wider student population in terms of age, experience level and other factors. Furthermore, the assignment given to the participants was relatively simple and straightforward; thus, there may be no way to extrapolate any findings from this evaluation process to more complex programming assignments. \par

Another issue is whether or not any changes observed in performance are attributable solely to the feedback system. As mentioned before, some students failed to submit solutions due to a lack of understanding or technical difficulties; such factors could have had an impact on their performance and should be accounted for when analyzing results. Additionally, the instructor's involvement in providing resources such as tutorial videos and sample solutions could have inadvertently influenced the students’ performance; thus, it would be advisable for future research projects to look into different approaches for providing assistance without having an effect on student performance. \par

Finally, the last limitation deals with how generalizable results from this study are for similar studies in different contexts. The participants in this study were from universities located all over Europe and Africa, so it would be difficult to draw conclusions from this study about how well a similar feedback system would perform at institutions located elsewhere. Furthermore, as noted earlier, some students found certain aspects of the assignment difficult or overly time-consuming; thus, any conclusions drawn from this study may not apply when dealing with more complex assignments at other institutions. \par

In conclusion, while this evaluation process has provided valuable insights into how well a feedback system can improve student programming performance through mutation testing techniques, further research is necessary before these findings can be generalized across different contexts.


\section{Conclusion}
The research on feedback systems has demonstrated that providing relevant and customizable feedback hints is a successful strategy for assisting early learners in learning and improving their JUnit testing skills over time. The evaluation showed that the structure of the customizable feedback hints was simple, which enabled the students to understand it easily. This implies that the system will so be great for lecturers who have to deal with a large number of students assignments. Although early learners are the target group, the system can also generate feedback for more experienced programmers because the feedback hints are customizable. \par

The feedback system was organized into categories and arranged into basic folders for easy understanding and scalability. Descriptive class and method names were used to enable the readability of the source code by other Java developers. This approach facilitates future expansions while ensuring clarity of the system design. \par

Tracking activity of students using the system revealed that those who got regular feedback hints submitted more solutions than those who did not get any hints. In addition, a comparison between the time taken by feedback systems and manual feedback showed a significant difference in favor of automated evaluation and giving of feedback. The data gathered from my feedback system suggests that it is indeed simple and provides meaningful feedback hints that aid students in improving their programming skills quickly. Furthermore, the feedback hints are customizable, making the system even more effective in helping students reach their desired proficiency level. \par

In conclusion, this research has demonstrated that providing relevant and customizable feedback hints is an effective strategy for assisting early learners in learning and improving their JUnit testing skills over time. The development of feedback systems opened the door for significant programming improvements in the way JUnit testing concepts were presented to early learners of OOP and how the customized feedback hints assisted them in sharpening their skills. \par

The use of feedback systems is generally perceived to be excellent in giving customized hints on JUnit tests; however, further research should look more closely at how historical information about student solutions may be used to enhance such suggestions as well as explore new technologies and designs used for creating future automated evaluation systems with multiple feedback hints per mutation testing process.

\section{Future Work}
Based on these findings, it is recommended that further analyses could be conducted on how automated feedback systems affect students at different levels or with different types of programming backgrounds before any definitive conclusions can be drawn about its effectiveness overall across all types of programming courses or assignments. I have tried to locate as many various references that will support my arguments as feasible to make my system as dependable as possible. My thesis' credibility is founded on present knowledge and actuality.\\
\\
 Additionally, future work shall involve adding a user interface so that the professor does not influence the feedback hints by adjusting the customizable feedback hints and thresholds more easily. The system shall also generate feedback hints that are auto-edited instead of having the professor pre-written on text files which is tiresome and time-consuming. 
% The results of this research indicate that providing customized feedback to early learners of Object-Oriented Programming is beneficial in several ways. Firstly, it can help them identify errors in their code programs more quickly and easily by providing customized feedback hints as to where their solutions may have gone wrong. Secondly, it provides insight into potential areas of improvement in their solutions, which can help them become better programmers in the long run. Lastly, utilizing mutation testing tools such as PiTest can help improve the test suites used to evaluate student solutions, leading to more accurate assessment results.
% \section{Interpretations}
% Looking at the customized feedback system, the results have shown that giving customized feedback to students is important as it helps the identify errors in a code and rectify them. In the first interaction of the system, the PITest tool has come out as an outstanding tool that is compatible with this system and it is accurate and efficient in generating mutants. Generating feedback based on tests is a useful way to point out errors in student solutions and emphasizes the importance of mutation testing to programming students. It is therefore a valuable technique, and relatively easy to implement using existing test frameworks. Most tools that use automated testing support the PITest tool, because black-box testing does not require using a specific algorithm.\par 
% The only aspect that matters is whether the output meets the requirements of the exercise. However, only giving test-based feedback will not in all cases help a student fix an incorrect program. To help a student improve their JUnit testing skills, the feedback given should explain why the test failed, how the code was supposed to be, and suggestions on how to re-edit the code. Feedback on how to proceed is necessary to both fix problems and progress toward a solution when stuck. This type of feedback is mostly seen in the form of error correction feedback hints and much less as hints on task-processing steps as well as program improvements.\par 
% The findings from this research suggest that providing customized feedback is beneficial for early learners of Object-Oriented Programming because it helps them identify their errors and fix them more quickly and easily and provides insight into potential areas for improvement in their solutions. Furthermore, utilizing mutation testing tools such as PiTest can help improve the test suites used to evaluate student solutions and lead to more accurate assessment results. Thus, this type of feedback system could be useful for helping students learn and grow as programmers over time.\par 
%  Tests suits that passed mutation testing means they are good and accurate. The correct test suites can be obtained and applied to programming and work precisely while delivering accurate and efficient results. In dynamic analysis, they are used for running test cases to generate the expected output. In static analysis, the structure of a correct solution is compared to the structure of a student solution. To recognize more than one solution variant, the PITest tool accepts multiple solutions that each represent a different algorithm to solve the problem. This enables multiple feedback to be given to students within a short time duration.\par 
% The system offers the professor several options to customize the hints using special commands embedded in the comments. Using these commands the professor can specify custom hint messages and control which parts of the code should be revealed or not. The system includes a default revealing mechanism that shows the basic structure of the code and ends at the level that shows all code.\par 
% There is a vast amount of studies that identify and classify difficulties and errors that novice programmers encounter in their learning process. In my case, the customized feedback system had higher levels of feedback satisfaction compared to generalized feedback. This study is also crucial in assessing the effectiveness of customized feedback in terms of both academic achievement and satisfaction with feedback compared to generalized feedback.
% \section{Implications}
% The implications of this study are significant because they suggest that providing meaningful and customizable feedback hints is an effective way to help beginners learn and develop programming skills over time with greater ease and accuracy. This type of system would be especially useful for instructors or tutors who are teaching large classes or dealing with a large volume of assignments regularly due to its automated nature which reduces the amount of time spent giving individualized feedback for each student's assignment solution.\par 
% Many of the previous feedback systems have only been able to deliver generalized feedback. However, my customized feedback system goes further to give a customized feedback hint that beginner students with basic JUnit testing skills can understand. If students want to learn from their mistakes, they should familiarize themselves with using my customized feedback system. The use of this system enables lecturers to turn their attention to the students whose code failed. By doing so they can engage the students physically and help them identify their mistakes and improve.
% \section{Limitations}
% Despite the promising findings from this study, some limitations should be noted when interpreting these results. It was discovered that customized feedback did not improve students' academic achievement, despite the precedence set by the literature. However, it was discovered that the students who received customized feedback expressed feedback satisfaction at much greater levels.
% Additionally, this study did not investigate whether or not this type of automated feedback system had any potential negative impact on student learning outcomes; thus further research should also be conducted to assess any potential drawbacks associated with using automated systems like these before they are widely implemented in classrooms or other educational settings around the world.\par 
%  Some students believed that the exercise's improper parametrization made it difficult to compute the system's inputs. Additionally, others thought that the task required laborious, mechanical processes that diminished the learning process and frequently resulted in dissatisfaction. I appreciate the validity of these critiques and admit that much more effort has to be done to improve both the content and presentation of future assignments in the customized feedback because this is the first actual deployment and testing of the customized feedback. 
% \section{Recommendations and future studies}
% Based on these findings, it is recommended that further studies or analyses be conducted on how automated feedback systems like these affect students at different levels or with different types of programming backgrounds before any definitive conclusions can be drawn about its effectiveness overall across all types of programming courses or assignments. I have tried to locate as many various references that will support my arguments as feasible to make my system as dependable as possible. My thesis' credibility is founded on present knowledge and actuality.\par 
%  Additionally, future work shall involve adding a user interface so that the professor does not influence the feedback hints by adjusting their thresholds therefore, the hits should entirely be auto-generated to deliver meaningful feedback to the students especially beginner students with basic knowledge. The system shall also generate feedback hints that are auto-edited instead of having the professor edit them manually which is tiresome and time-consuming. Although the resulting feedback hits might seem robotic rather than manual because of limited human intervention, the system shall been customized to deliver feedback hits that will be as good as manual.\par 
% The diversity of feedback types has increased over the last decades and new techniques are being applied to generate feedback that is increasingly helpful for students. Therefore, if a similar study were to be conducted at a different time, it's feasible that the results would be different. Given that my thesis deals with customized feedback, a relatively new concept, significant modifications may be made in one or more of these fields.
% The most frequent types of feedback are still test failures and solving problems when taking into account past data. The diversity of feedback styles has, however, grown in the twenty-first century, which is a good thing.

%\section{Summary: Recap of key results}
%The results of this research indicate that providing customized feedback to early learners of Object-Oriented Programming is beneficial in several ways. Firstly, it can help them debug their programs more quickly and easily by providing specific hints as to where their solutions may have gone wrong. Secondly, it provides insight into potential areas of improvement in their solutions, which can help them become better programmers in the long run. Lastly, utilizing mutation testing tools such as PiTest can help improve the test suites used to evaluate student solutions, leading to more accurate assessment results.
%\section{Interpretations: Meaning behind the results}
%The findings from this research suggest that providing customized feedback is beneficial for early learners of Object-Oriented Programming because it helps them debug their programs more quickly and easily and provides insight into potential areas for improvement in their solutions. Furthermore, utilizing mutation testing tools such as PiTest can help improve the test suites used to evaluate student solutions and lead to more accurate assessment results. Thus, this type of feedback system could be useful for helping students learn and grow as programmers over time. 
%\section{Implications: Significance of the results}
%The implications of this study are significant because they suggest that providing meaningful and customizable feedback hints is an effective way to help beginners learn and develop programming skills over time with greater ease and accuracy. This type of system would be especially useful for instructors or tutors who are teaching large classes or dealing with a large volume of assignments regularly due to its automated nature which reduces the amount of time spent giving individualized feedback for each student's assignment solution.
%\section{Limitations: What cannot be deduced from the results?}
%Despite the promising findings from this study, some limitations should be noted when interpreting these results. Firstly, this study only focused on early learners with basic knowledge of JUnit testing; thus further research should be conducted on how this type of feedback system affects students at different levels or with different types of programming backgrounds before any definitive conclusions can be drawn about its effectiveness overall across all types of programming courses or assignments. Additionally, this study did not investigate whether or not this type of automated feedback system had any potential negative impact on student learning outcomes; thus further research should also be conducted to assess any potential drawbacks associated with using automated systems like these before they are widely implemented in classrooms or other educational settings around the world.  
%\section{Recommendations: Further studies or analyses}
%Based on these findings, it is recommended that further studies or analyses be conducted on how automated feedback systems like these affect students at different levels or with different types of programming backgrounds before any definitive conclusions can be drawn about its effectiveness overall across all types of programming courses or assignments. Additionally, further research should also be conducted to assess any potential drawbacks associated with using automated systems like these before they are widely implemented in classrooms or other educational settings around the world. Finally, further research should also be conducted on how to optimize the feedback provided by these systems to maximize their effectiveness in helping students learn and develop programming skills over time.

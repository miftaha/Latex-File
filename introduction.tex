\chapter{Introduction}
\label{chapter:introduction}
\section{Overview}
Software testing was formerly undervalued because of a variety of issues, including costs and the demand on resources. Software testing has, nevertheless, become more popular recently.
As they give students the chance to get real-world experience, programming assignments are crucial to the teaching of computer science. 

When a large number of students are engaged in programming assignments, obtaining feedback from them on their work can be difficult for professors to manage. While grades only provide students a basic grasp of how their work compares against the standards, the newly developed customized feedback system that uses a  mutation testing approach makes it easier for students to examine test cases and evaluate the quality of these tests automatically. \\

Programming assignments are essential to the teaching of computer science as they provide students with the opportunity to practice what they have learnt in school and gain practical experience. However, giving students feedback on their programming tasks can be a difficult task for teachers, especially when there are many students involved. On the other hand, grades merely provide students with surface-level insight into how their work compares to the criteria that should be in place for every specific assignment. To address this problem, one possible solution is to utilize mutation testing approach, a method that will examine and evaluate the quality of the tests suite automatically. \par

Given the importance of feedback in promoting learning and increasing students’ knowledge, the development of this feedback system aimed to provide students with meaningful customizable feedback hints to enable them to spot errors in their submitted assignment solutions as well as get suggestions on how to fix those errors. \par 

Students’ feedback on their programming assignments can be challenging for professors when a large number of students are engaged. This problem can be solved by the development of an automated feedback system that uses mutation testing process to evaluate tests cases. \par

In order to fully automate this process, it is necessary to synchronize a Dropbox folder containing student submissions with a local project folder using a program such as Rclone. Additionally, a library such as Quartz can be used to schedule regular scans and executions of the test projects submitted by students. This would ensure that feedback is generated in an automated fashion without any manual intervention required.\\
 
The purpose of this study is twofold: (1) explore what type of auto-generated feedback hints are useful for early learners of object oriented programming with basic knowledge of Unit Testing; (2) investigate how we can customize this proposed feedback system to automatically generate meaningful feedback hints for such learners \par

Tools that help students learn programming have been in use since the 1960s \cite{ref1}.
 These tools help students make use of programming knowledge that helps them understand how a program works \cite{ref2}. Programming is difficult to learn \cite{ref3} and students need assistance to make progress \cite{ref4}. This is because tens of thousands of students globally \cite{ref5} take programming courses, hence teaching and evaluating individual students can be time-consuming and tiresome for professors \cite{ref6}.\par
 
Feedback is an essential element of learning \cite{ref7,ref8}. Boud and Molloy defined feedback as "the process by which learners obtain information about their work to recognize the differences and similarities between the standards required for any task given and the qualities of the work itself to produce improved work" \cite{ref9}. Based on this definition, formative feedback is "information conveyed to the learner with the intention to modify his or her thinking or behavior for the purpose of better learning" \cite{ref10}. Learners can also gain some insight into their performance through summative feedback in the form of marks or percentages on examinations \cite{ref11}.  On the other hand, grades merely provide students with surface-level insight on their work compared to the criteria that should be in place for every specific assignment. \par 

To evaluate the quality of student-submitted JUnit solutions, a mutation testing process will be carried out with the help of Pitest mutation testing tool. The students will submit their solutions to the provided "projects submitted" folder in a Dropbox and receive meaningful customized feedback hints automatically via their respective email addresses. The professor or tutor should be able to customize the Mutation Threshold score of the feedback system to indicate whether the student test suit is good or bad at first glance. PITest is a powerful tool for generating feedback for programming assignments. It works by introducing small changes (called mutations) to the code and then evaluating whether those changes were detected by the unit tests. If the changes were not detected, it means that the unit test coverage could be improved and we can generate feedback accordingly. In order to customize the feedback message or hints, the feedback system need to read through the mutation testing report that is generated after executing Pitest. This report will contain information about which mutants survived and which were killed by the unit tests. The feedback system will extract this information from the report into a SurvivedMutant model class attribute, which will provide detailed information about each mutation such as its location in the code and any other relevant details. With this information, we can create meaningful feedback messages or hints tailored to each student's programming assignment. These messages can include examples of how to kill certain mutants as well as general tips for improving their unit test coverage. By taking this approach, we can ensure that each student receives valuable feedback on their assignment that is specific to their own codebase which they can use to improve their programming skills. \\

This thesis will begin by providing a review of related resources on gaining programming knowledge in Section 2. Section 3 details my study topics and methodology while Section 4 presents the evaluation and explores the labelling process. The outcomes are discussed in Section 5, before the paper is concluded with future endeavors outlined in Section 6. 

% Programming assignments are essential to the teaching of computer science because they give students the chance to practice what they have learnt and gain practical experience. However, giving students feedback on their programming tasks can be difficult for teachers, especially when there are many students involved. Utilizing mutation testing approach, a method that may produce test cases and assess the caliber of the tests automatically is one possible response to this issue.\par 
% Since the 1960s \cite{ref1}, tools that supports students in learning programming have been developed.  These tools provide a simplified programming environment, utilize animation or visualization to provide a better understanding of how a program runs, guide students to the correct program via hints and feedback messages, and automatically grade student solutions \cite{ref2}. Programming is difficult to learn \cite{ref3}, and students need assistance to make progress \cite{ref4}; programming courses are taken by tens of thousands of students across the globe \cite{ref5}, and assisting individual student with their problems requires a significant time commitment from teachers \cite{ref6}.\par 
% Feedback is an essential element of learning \cite{ref7,ref8}. Boud and Molloy define feedback as "the process by which learners obtain information about their work in order to recognize the differences and similarities between the standards required for any task given and the qualities of the work itself in order to produce improved work" \cite{ref9}. By this definition, formative feedback is "information conveyed to the learner with the intention to modify his or her thinking or behavior for the purpose of better learning" \cite{ref10}. A learner can also gain some insight into their performance through summative feedback in the form of marks or percentages on examinations \cite{ref11}.  On the other hand, grades merely provide students with surface-level insight into how their work compares to the criteria that should be in place for every specific assignment. \par 
% This master's thesis presents the development of a fully automated feedback system to provide meaningful customized auto-generated feedback hints to early learners of Object-Oriented Programming with basic knowledge of JUnit testing. To evaluate the quality of student submitted JUnit tests assignment solutions, mutation testing process was utilized. Students will be able to submit their solutions and receive helpful feedback hints that are more understandable compared to those generated by PiTest mutation tool. \par 
% The meaningful feedback hints will be provided to students directly to students’ provided email addresses. The professor or tutor should be able to customize the Mutation Threshold score of the feedback system, which will indicate that student test suit is at the first instance good or bad. \par 
% Given the importance of feedback in education, I am interested in exploring what type of auto-generated feedback hints are useful for early learners of OOP with basic knowledge of Unit Testing and How can the proposed feedback system be fully customized to generate meaningful feedback hints to early learners of OOP with basic knowledge of JUnit testing? \par 
% This thesis will begin by providing a review of related resources on gaining programming knowledge in Section 2. Section 3 details my study topics and methodology while Section 4 presents the results and explores the labelling process. The outcomes are discussed in Section 5, before the paper is concluded with future endeavors outlined in Section 6.
\section{Features and Challenges} 
The feedback system is fully automated and the feedback hints are also fully customizable. It is integrated with Dropbox, which is used to store student’s projects. It has notification feature to notify the student about testing result via student’s email. It also automatically scans student’s submitted projects, run JUnit test and send email notification, thereby making it fully automated. Additionally, it also provides to student an example of mutation sample code on how possibly survived mutants can be killed.\newline

The System is designed to move files from a cloud storage service to local storage, which can be configured by time schedule. This enables the system to scan student-submitted projects in the Dropbox folder named “project submitted” and move them to the local storage folder called “projects submitted” for execution. After the projects have been scanned and executed, they will be moved back to the Dropbox folder depending on the test results.\par 

In addition, the system will execute JUnit tests and Pi-tests to verify whether a student's Junit test cases are passed or not, and compare it with the professor's pre-configured threshold mark. Furthermore, feedback hints will also be provided to students on their JUnit test cases that failed, as well as how they can improve their solutions. Lastly, an email notification will be sent to each student via their provided email address with content based on the JUnit test cases result.\newline
 

The challenge of providing a means to fully customize the feedback hints for students requires developing an automated feedback system that can be easily customized by professor or tutor. This means creating a means that allows professor or tutor to write custom messages such as what types of feedback they want to give to students and how detailed it should be. In order to fully automate the feedback system, an automated system must be developed that can automatically synchronize Dropbox folder and Local folder using Rclone program. Furthermore, this system needs to be able to automatically scan and execute students submitted test solutions using timer schedule provided by Quarzt library.\par

Providing more helpful and easy-to-understand feedback hints to students requires figuring out a way to allow the professor or tutor to be able to write custom feedback hints for students. Moreover, it should be possible to provide students with code examples in order to support them in improving their test cases.  Finally, the PiTest generated feedback needs to be read in order for the professor or tutor’s customized feedback hints for students to be effective. This requires designing algorithms that can read through the PiTest auto-generated results and extract relevant information such as which mutants survived or was killed.



\section {Problem Statement}  
More than ever, contributions to the field of software development are on the rise. Researchers have previously utilized mutation testing to artificially introduce faults into software to test its robustness.The problem addressed in this research is that many programming assignments do not receive adequate feedback. This is often due to the fact that grading is done manually, and this can be difficult for instructors to provide detailed feedback for every student especially with the fact that a single problem can be described with different algorithms and the same algorithm can be implemented in a number of different ways hence burdening the grading process. Moreover, automated assessment systems are often limited in their ability to provide meaningful feedback hints to the students. As a result, it is difficult for students to identify and correct their mistakes in order to improve their programming skills. 
\par 
This research will seek to develop a feedback system that provides meaningful and customizable auto-generated feedback hints to early learners of Object-Oriented Programming (OOP) with basic knowledge of JUnit testing. The proposed system will utilize PiTest a mutation testing tool to evalaute the robustness of the student's submitted assignments. Furthermore, the feedback hint generated by this system will be easy for students to understand, and they will be sent directly to the student’s provided E-Mail address. 
\par 
Finally, this research will explore how useful the customized feedback hints are for early learners while considering how applicable these customized auto-generated feedback hints can enhance early learners' JUnit testing skills.
\section{Research questions}
The contribution to the field of software development has been rising more than ever. Mutation testing is a process of artificially introducing faults into a program and then testing the program to see if the faults are detected or not. If the faults are not detected, then the test suite is said to be not robust and feedback will be generated to the user. \par 

In order to answer the research questions, this thesis will focus on the development of a feedback system that can provide customized auto generated feedback hints to early learners of Object-Oriented Programming (OOP) with basic knowledge of JUnit testing. \par
This system can be used to evaluate the quality of students’ submitted JUnit tests assignment solutions by utilizing mutation testing. The aim is to investigate how useful such customizable feedback hints are for early learners of object oriented programming students with basic knowledge of JUnit testing and how they can enhance their JUnit testing skills.\newline \newline
\begin{enumerate}
\item What type of auto-generated feedback hints is useful for early learners of OOP with basic knowledge of Unit Testing? \par 

For early learners of OOP with basic knowledge of JUnit testing, it is very important to provide feedback that encourages exploration and experimentation of JUnit testing. This can be achieved by providing helpful hints about the error in their code, as well as suggestions on how to improve the test cases. Additionally, providing example solutions or snippets of code which demonstrate best practices can help learners understand the concepts more clearly.  

\item How can the proposed feedback system get fully automated to generate meaningful feedback hints to early learners of OOP with basic knowledge of JUnit testing?\par 

To fully automate the proposed feedback system, it is necessary to utilize an automated synchronization tool such as Rclone which can synchronize Dropbox folder and Local folder. Additionally, an execution timer schedule provided by Quartz library should be set up so that submitted solutions are automatically scanned and executed at regular intervals. Furthermore, an automated system that makes use of PI-Test should be implemented to generate meaningful feedback hints by simulating errors in the code and checking if students’ tests are able to identify them or not. By combining these components, a fully automated system will be able to provide meaningful feedback hints to students. This will help them learn how to identify bugs in their own programs and write more effective tests. 


\end{enumerate}

\section{Motivation}
The process of developing software must include software testing in order to ensure the quality and reliability of the software. Studies have shown that software testing can cost up to 50\% of the budget for development. Software tests are used to determine how well developers understand a program’s limitations and potential problems as well as how it should operate. Test Driven Development (TDD) is an approach used in the current development cycle with the goal of creating “well known” software that is thoroughly tested in nearly every conceivable aspect. Code coverage is a measure used to track how much of our code is covered in tests, providing information on the written source code’s many elements such as conditionals coverage, line coverage, and others.\par 
Despite having high code coverage, we frequently encounter mistakes, bugs, and other problems in production. This raises questions about the effectiveness of our tests; are they sufficient? How many tests do we have? Are all the edge scenarios where our program fails being tested? These questions can be answered through mutation testing. Mutation testing was first proposed over 40 years ago and entails putting changed versions of the source code through software testing. By using mutation operators, modified copies are produced which should fail when tested due to their altered source code. This can help us improve our test suites by better identifying bugs and errors present in our programs.

%\noindent\lipsum[4-10]

